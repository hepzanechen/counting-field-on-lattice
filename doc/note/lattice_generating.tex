%% LyX 2.4.2.1 created this file.  For more info, see https://www.lyx.org/.
%% Do not edit unless you really know what you are doing.
\documentclass[english]{article}
\usepackage[LGR,T1]{fontenc}
\usepackage{textcomp}
\usepackage[latin9]{inputenc}
\usepackage{xcolor}
\usepackage{float}
\usepackage{amsmath}
\usepackage{amssymb}
\usepackage{graphicx}
\usepackage{geometry}
\geometry{verbose,lmargin=1.5cm,rmargin=1.5cm}

\makeatletter

%%%%%%%%%%%%%%%%%%%%%%%%%%%%%% LyX specific LaTeX commands.
\DeclareRobustCommand{\greektext}{%
  \fontencoding{LGR}\selectfont\def\encodingdefault{LGR}}
\DeclareRobustCommand{\textgreek}[1]{\leavevmode{\greektext #1}}

\floatstyle{ruled}
\newfloat{algorithm}{tbp}{loa}
\providecommand{\algorithmname}{Algorithm}
\floatname{algorithm}{\protect\algorithmname}

\makeatother

\usepackage{babel}
\begin{document}
\title{Counting Fields in Tight Binding Lattice}
\author{Zhen Chen}
\maketitle

\section{Motivation}

The advantage of using Green's functions lies in their ability to
simplify the computation of certain basic quantities, such as the
density of states (DOS) and transmission, which can be expressed in
terms of lower-order Green's functions. However, when it comes to
calculating multi-particle correlators, expressing them using single-particle
Green's functions can become cumbersome and lengthy. In such cases,
the Counting Field method offers a significant advantage.

Since the pioneering work in Ref. {[}4{]}, the Counting Field method
has been applied to quantum dot transport {[}3{]}, thermal transport
{[}4{]}, and other cases.However, these studies have largely focused
on the analytical capabilities of the Counting Field method. In this
article, we extend the application of the Counting Field method to
computational transport within tight-binding models.

Specifically, we introduce the counting field into the hopping terms
of a tight-binding lattice, calculate the action via the determinant
operation, and determine currents and current correlations through
variations with respect to the counting field. When compared to the
conventional recursive Green\textquoteright s function method used
in tight-binding models, the Counting Field method demonstrates comparable
computational efficiency. Moreover, it offers an advantage in its
convenience for calculating higher-order correlations.

\section{Therotical Backgrounds.}


\subsection{Workflow}

\begin{figure}[H]
\begin{centering}

\par\end{centering}
\caption{The path integral contour}

\end{figure}

We begin by considering a general action describing a tight-binding
model on a time contour:

\[
S_{0}=\sum_{mn}\left\{ \int dt\bar{\phi}_{m}^{+}(t)\left[\delta_{mn}i\partial_{t}\phi_{n}^{+}(t)-{\color{teal}e^{i\frac{\lambda_{mn}(t)}{2}}}\,t_{mn}\phi_{n}^{+}(t)\right]-\int dt\bar{\phi}_{m}^{-}(t)\left[\delta_{mn}i\partial_{t}\phi_{n}^{-}(t)-{\color{teal}e^{-i\frac{\lambda_{mn}(t)}{2}}}t_{mn}\phi_{n}^{-}(t)\right]\right\} 
\]
Here, $m$ and $n$ are indices for lattice sites. $\phi_{m}^{+}(t)$ and $\phi_{m}^{-}(t)$ are the Grassmann fields residing on the forward (+) and backward (-) branches of the Keldysh time contour, respectively. The counting field $\lambda_{mn}(t)$ is associated with the hopping from site $n$ to site $m$ and satisfies $\lambda_{mn}(t) = -\lambda_{nm}(t)$. 

The expectation value of the current operator $\left\langle I_{mn}(t)\right\rangle$ is obtained through the functional derivative of the action with respect to the counting field:

\[
\left\langle I_{mn}(t)\right\rangle = \dfrac{\delta\ln(S_{0})}{\delta(i\lambda(t))}\bigg|_{\lambda(t)=0}
\]

Similarly, the current-current correlation function $\left\langle I_{mn}(t)I_{mn}(t^{\prime})\right\rangle$ is given by the second-order functional derivative:

\[
\left\langle I_{mn}(t)I_{mn}(t^{\prime})\right\rangle = \dfrac{\delta^{2}\ln(S_{0})}{\delta(i\lambda(t))\delta(i\lambda(t^{\prime}))}\bigg|_{\lambda(t)=0}
\]

\begin{algorithm}[H]
\begin{centering}

\par\end{centering}
\caption{Comparing the Counting Field method for calculating observables with
the recursive Green's function, the major change in labor is from
matrix inversion to matrix determinant calculation.}
\end{algorithm}

To facilitate the computation, we transform the action into the frequency domain via Fourier transformation. We assume time-translation invariance, under which the current $I_{mn}(t)$ becomes independent of the specific time $t$. This implies that the corresponding counting field $\lambda$ can be treated as constant in the frequency domain. 
Subsequently, a Keldysh rotation is applied to decompose the action into its standard retarded, advanced, and Keldysh components.

Evaluating the path integral using the Grassmann Gaussian integral formalism yields the generating functional $Z(\boldsymbol{\lambda}, \omega)$, which is found to be proportional to the determinant of a matrix $\mathcal{M}(\boldsymbol{\lambda},\omega)$:
\[
Z(\boldsymbol{\lambda},\omega) = \mathcal{N} \det\mathcal{M}(\boldsymbol{\lambda},\omega).
\]
Here, $\mathcal{N}$ represents a normalization factor that depends on the system's equilibrium distribution 
but is independent of the counting fields $\boldsymbol{\lambda}$. 
Observables such as current and noise correlations are subsequently obtained via functional derivatives of $\ln Z(\boldsymbol{\lambda}, \omega)$ with respect to $\boldsymbol{\lambda}$.

where,$\boldsymbol{\lambda}=\left(\lambda_{1},\cdots,\lambda_{N_{L}}\right)$
,$\lambda_{i}$ is generating fields for $i$th lead, $N_{L}$ is
total lead number , the explict form of $\mathcal{M}$ matrix is 

The vector $\boldsymbol{\lambda}=\left(\lambda_{1},\cdots,\lambda_{N_{L}}\right)$
consists of generating fields for each lead $i$, with $N_{L}$ being
the total number of leads. The explicit form of the $\mathcal{M}$
matrix is:

\[
\left.\mathcal{M}(\boldsymbol{\lambda},\omega)=\left(\begin{array}{cccc}
\boldsymbol{g_{Central}^{-1}(\omega)} & \boldsymbol{-t_{CL1}\Lambda_{1}(\lambda_{1},\omega)} & \boldsymbol{\cdots} & \boldsymbol{-t_{CLN_{L}}\Lambda_{N_{L}}(\lambda_{N_{L}},\omega)}\\
\boldsymbol{-\Lambda_{1}^{\dagger}(\lambda_{1},\omega)t_{CL1}^{\dagger}} & \boldsymbol{g_{Lead1}^{-1}(\omega)} & \boldsymbol{} & \boldsymbol{}\\
\boldsymbol{\vdots} & \boldsymbol{} & \boldsymbol{\ddots} & \boldsymbol{}\\
\boldsymbol{-\Lambda_{N_{L}}^{\dagger}(\lambda_{N_{L}},\omega)t_{CLN_{L}}^{\dagger}} & \boldsymbol{} & \boldsymbol{} & \boldsymbol{g_{LeadN_{L}}^{-1}(\omega)}
\end{array}\right.\right)
\]

Each element in the matrix has a 2\texttimes 2 block structure:
\[
\boldsymbol{g_{c0}^{RKA}}=\left(\begin{array}{cc}
\left(g_{C}^{r}\right)^{-1} & \left(g_{C}^{r}\right)^{-1}g_{C}^{k}\left(g_{C}^{a}\right)^{-1}\\
 & \left(g_{C}^{a}\right)^{-1}
\end{array}\right)
\]

\[
\boldsymbol{t_{CL}}=\left(\begin{array}{cc}
t_{CL}\\
 & t_{CL}
\end{array}\right);
\]
\[
\boldsymbol{\Lambda_{1}(\lambda_{1},\omega)}=\left(\begin{array}{cc}
\cos\left(\dfrac{\lambda}{2}\right) & -i\sin\left(\dfrac{\lambda}{2}\right)\\
-i\sin\left(\dfrac{\lambda}{2}\right) & \cos\left(\dfrac{\lambda}{2}\right)
\end{array}\right)
\]

For instance, when there is only one lead, the matrix $\mathcal{M}(\boldsymbol{\lambda},\omega)$
\begin{equation}
\mathcal{M}(\boldsymbol{\lambda},\omega)=\left(\begin{array}{cccc}
\left(g_{C}^{r}\right)^{-1} & \left(g_{C}^{r}\right)^{-1}g_{C}^{k}\left(g_{C}^{a}\right)^{-1} & t_{LC}\cos\left(\dfrac{\lambda}{2}\right) & -it_{LC}\sin\left(\dfrac{\lambda}{2}\right)\\
 & \left(g_{C}^{a}\right)^{-1} & -it_{LC}\sin\left(\dfrac{\lambda}{2}\right) & t_{LC}\cos\left(\dfrac{\lambda}{2}\right)\\
t_{CL}^{*}\cos\left(\dfrac{\lambda}{2}\right) & it_{LC}^{*}\sin\left(\dfrac{\lambda}{2}\right) & \left(g_{Leadi}^{r}\right)^{-1} & \left(g_{Leadi}^{r}\right)^{-1}g_{Leadi}^{k}\left(g_{Leadi}^{a}\right)^{-1}\\
it_{LC}^{*}\sin\left(\dfrac{\lambda}{2}\right) & t_{CL}^{*}\cos\left(\dfrac{\lambda}{2}\right) &  & \left(g_{Leadi}^{a}\right)^{-1}
\end{array}\right)\label{eq:M explicty}
\end{equation}

Note that each block is also multi-dimensional; for example, $\left(g_{C}^{r}\right)^{-1}$
has dimensions $N_{C}\times N_{C}$, where $N_{C}$ is the size of
the central region.

In practical calculations, the final non-equilibrium state of the
central region is entirely determined by the leads, so we can drop
the term $\left(g_{C}^{r}\right)^{-1}g_{C}^{k}\left(g_{C}^{a}\right)^{-1}$
in the matrix \ref{eq:M explicty}.. This simplification allows $\left(g_{C}^{r,a}\right)^{-1}$
to be expressed simply as $\omega-H_{C}\pm i\eta.$

Compare to old Recursive Green's function method, now we only need
to do determinant calcuation. The time complexity for both matrix
inversion and determinant calculation for band matrices, to which
Hamiltonian matrices often belong, is essentially equivalent, typically
of the order $O(MN^{3})$, where $N$ is the dimension of the small
matrix, and $M$ is the number of blocks.Determinant calculation generally
has a lower space complexity since it can often be performed in-place
or with minimal additional storage.


\subsection{Analytically verication }

In this section, we analytically calculate $\left\langle I_{mn}(t)\right\rangle =i\dfrac{\delta\ln(S_{0})}{\delta\lambda(t)}$
using the generating function method, in order to verify that it corresponds
to the familiar current formula.

By repeatedly applying the determinant identity $\left.\det\left(\begin{array}{cc}
A & B\\
C & D
\end{array}\right.\right)=\det(D)\det\left(A-BD^{-1}C\right),$ we get:

\begin{equation}
\begin{aligned}\det\mathcal{M}(\boldsymbol{\lambda},\omega) & =\prod_{i=1}^{N_{L}}\det\left(g_{Leadi}^{-1}(\omega)\right)\det\left(g_{Central}^{-1}(\omega)-\sum_{i=1}^{N_{L}}t_{CLi}\Lambda_{i}(\lambda_{i},\omega)g_{Leadi}^{-1}(\omega)\Lambda_{i}^{\dagger}(\lambda_{i},\omega)t_{CLi}^{\dagger}\right)\\
 & =\prod_{i=1}^{N_{L}}\det\left(g_{Leadi}^{-1}(\omega)\right)\det\left(g_{Central}^{-1}(\omega)-\sum_{i=1}^{N_{L}}\Sigma_{i}\left(\lambda_{i},\omega\right)\right)
\end{aligned}
\end{equation}

For simplicity, let us consider$\boldsymbol{\lambda}=\left(\lambda_{1},0,\cdots,0\right)$

\begin{equation}
\begin{aligned}\dfrac{\ln Z(\lambda_{1},\omega)-\ln Z(0,\omega)}{\lambda_{1}} & =\dfrac{1}{\lambda_{1}}\ln\dfrac{\mathcal{N}\det\mathcal{M}(\lambda_{1},\omega)}{\mathcal{N}\det\mathcal{M}(0,\omega)}=\dfrac{1}{\lambda_{1}}\ln\dfrac{\prod_{i=1}^{N_{L}}\det\left(g_{Leadi}^{-1}(\omega)\right)\det\left(g_{Central}^{-1}(\omega)-\sum_{i=1}^{N_{L}}\Sigma_{i}\left(\lambda_{1}\delta_{i1},\omega\right)\right)}{\prod_{i=1}^{N_{L}}\det\left(g_{Leadi}^{-1}(\omega)\right)\det\left(g_{Central}^{-1}(\omega)-\sum_{i=1}^{N_{L}}\Sigma_{i}\left(0,\omega\right)\right)}\\
 & =\dfrac{1}{\lambda_{1}}\ln\dfrac{\det\left(g_{Central}^{-1}(\omega)-\sum_{i=1}^{N_{L}}\Sigma_{i}\left(0,\omega\right)+\Sigma_{1}\left(0,\omega\right)-\Sigma_{1}\left(\lambda_{1},\omega\right)\right)}{\det\left(\underset{G_{Central}^{-1}(\omega)}{\underbrace{g_{Central}^{-1}(\omega)-\sum_{i=1}^{N_{L}}\Sigma_{i}\left(0,\omega\right)}}\right)}\\
 & =\dfrac{1}{\lambda_{1}}\ln\det\left(\mathbb{I}+G_{Central}(\omega)\left(\Sigma_{1}\left(0,\omega\right)-\Sigma_{1}\left(\lambda_{1},\omega\right)\right)\right)
\end{aligned}
\end{equation}

Taking the limit as $\lambda_{1}\rightarrow0:$

\begin{equation}
\begin{aligned}\lim_{\lambda_{1}\rightarrow0}\dfrac{\ln Z(\lambda_{1},\omega)-\ln Z(0,\omega)}{\lambda_{1}} & =\lim_{\lambda_{1}\rightarrow0}\dfrac{1}{\lambda_{1}}\ln\det\left(\mathbb{I}+G_{Central}(\omega)\left(\Sigma_{1}\left(0,\omega\right)-\Sigma_{1}\left(\lambda_{1},\omega\right)\right)\right)\\
 & =\lim_{\lambda_{1}\rightarrow0}\dfrac{1}{\lambda_{1}}\text{Tr}\ln\left(\mathbb{I}+G_{Central}(\omega)\left(\Sigma_{1}\left(0,\omega\right)-\Sigma_{1}\left(\lambda_{1},\omega\right)\right)\right)\\
 & =\lim_{\lambda_{1}\rightarrow0}\dfrac{1}{\lambda_{1}}\text{Tr}\ln\left(\mathbb{I}+G_{Central}(\omega)\left.\left(\Sigma_{1}\left(0,\omega\right)-\Sigma_{1}\left(0_{1},\omega\right)\right.\right.\right.\\
 & \left.\left.\left.-\frac{i\lambda_{1}}{2}\sigma_{x}^{RAK}\Sigma_{1}\left(0_{1},\omega\right)+\frac{i\lambda_{1}}{2}\Sigma_{1}\left(0_{1},\omega\right)\sigma_{x}^{RAK}+O(\lambda_{1})\right)\right)\right)\\
 & =\lim_{\lambda_{1}\rightarrow0}\dfrac{1}{\lambda_{1}}\text{Tr}\ln\left(\mathbb{I}+G_{Central}(\omega)\frac{i\lambda_{1}}{2}\left(\Sigma_{1}\left(0_{1},\omega\right)\sigma_{x}^{RAK}-\sigma_{x}^{RAK}\Sigma_{1}\left(0_{1},\omega\right)\right)\right)\\
 & =\dfrac{i}{2}\text{Tr}\left[G_{Central}(\omega)\left(\Sigma_{1}\left(0_{1},\omega\right)\sigma_{x}^{RAK}-\sigma_{x}^{RAK}\Sigma_{1}\left(0_{1},\omega\right)\right)\right]
\end{aligned}
\end{equation}

In the second but last step we used $\ln\left(\mathbb{I}+tA\right)=tA+O(t)$

\begin{equation}
\begin{aligned} & \dfrac{i}{2}\text{Tr}\left[G_{Central}(\omega)\left(\Sigma_{1}\left(0_{1},\omega\right)\sigma_{x}^{RAK}-\sigma_{x}^{RAK}\Sigma_{1}\left(0_{1},\omega\right)\right)\right]\\
= & \dfrac{i}{2}\text{Tr}\left(\left.\left(\begin{array}{cc}
 & G^{a}(\omega)\\
G^{r}(\omega) & G^{k}(\omega)
\end{array}\right.\right)\left.\left(\begin{array}{cc}
\Sigma_{1}^{r}(\omega) & \Sigma_{1}^{k}(\omega)\\
0 & \Sigma_{1}^{a}(\omega)
\end{array}\right.\right)-\left.\left(\begin{array}{cc}
G^{r}(\omega) & G^{k}(\omega)\\
 & G^{a}(\omega)
\end{array}\right.\right)\left.\left(\begin{array}{cc}
0 & \Sigma_{1}^{a}(\omega)\\
\Sigma_{1}^{r}(\omega) & \Sigma_{1}^{k}(\omega)
\end{array}\right.\right)\right)\\
= & \dfrac{i}{2}\text{Tr}\left(G^{r}(\omega)\Sigma_{1}^{k}(\omega)+G^{k}(\omega)\Sigma_{1}^{a}(\omega)-G^{k}(\omega)\Sigma_{1}^{r}(\omega)-G^{a}(\omega)\Sigma_{1}^{k}(\omega)\right)\\
= & \dfrac{i}{2}\text{Tr}\left(\left(G^{r}(\omega)-G^{a}(\omega)\right)\Sigma_{1}^{k}(\omega)+G^{k}(\omega)\left(\Sigma_{1}^{a}(\omega)-\Sigma_{1}^{r}(\omega)\right)\right)\\
= & \text{\ensuremath{\dfrac{i}{2}}Tr}\left(\left(G^{r}(\omega)-G^{a}(\omega)\right)\left(\Sigma_{1}^{<}(\omega)+\Sigma_{1}^{>}(\omega)\right)+\left(G^{<}(\omega)+G^{>}(\omega)\right)\left(\Sigma_{1}^{a}(\omega)-\Sigma_{1}^{r}(\omega)\right)\right)\\
= & \dfrac{i}{2}\text{Tr}\left(\left(G^{r}(\omega)-G^{a}(\omega)\right)\left(2\Sigma_{1}^{<}(\omega)+\Sigma_{1}^{r}(\omega)-\Sigma_{1}^{a}(\omega)\right)\right.\\
 & \left.+\left(2G^{<}(\omega)+G^{r}(\omega)-G^{a}(\omega)\right)\left(\Sigma_{1}^{a}(\omega)-\Sigma_{1}^{r}(\omega)\right)\right)\\
= & i\text{Tr}\left(\left(G^{r}(\omega)-G^{a}(\omega)\right)\Sigma_{1}^{<}(\omega)+G^{<}(\omega)\left(\Sigma_{1}^{a}(\omega)-\Sigma_{1}^{r}(\omega)\right)\right)\\
\\\end{aligned}
\label{eq:current 5}
\end{equation}

Notice that in the third line after applying the trace over the RAK
space, the remaining trace is taken over lattice sites, etc.

In the middle step we used $G^{>}(\omega)-G^{<}(\omega)=G^{r}(\omega)-G^{a}(\omega)$.

The current calculated from the Counting Field method, as derived
in Eq.\ref{eq:current 5} , is in agreement with the common current
formula:

\[
I=\frac{e}{\hbar}\int\frac{d\omega}{2\pi}\text{Tr}\left[\left(\mathbf{G}(\omega)\boldsymbol{\Sigma}_{1}(\omega)-\boldsymbol{\Sigma}_{1}(\omega)\mathbf{G}(\omega)\right)^{<}\right]
\]

\[
.\begin{aligned}I & =\frac{e}{\hbar}\int\frac{d\omega}{2\pi}\text{Tr}\left[\mathbf{G}^{r}(\omega)\boldsymbol{\Sigma}_{1}^{<}(\omega)+\mathbf{G}^{<}(\omega)\boldsymbol{\Sigma}_{1}^{a}(\omega)-\boldsymbol{\Sigma}_{1}^{r}(\omega)\mathbf{G}^{<}(\omega)-\boldsymbol{\Sigma}_{1}^{<}(\omega)\mathbf{G}^{a}(\omega)\right]\\
 & =\frac{e}{\hbar}\int\frac{d\omega}{2\pi}\text{Tr}\left[\left(\mathbf{G}^{r}(\omega)-\mathbf{G}^{a}(\omega)\right)\boldsymbol{\Sigma}_{1}^{<}(\omega)+\mathbf{G}^{<}(\omega)\left(\boldsymbol{\Sigma}_{1}^{a}(\omega)-\boldsymbol{\Sigma}_{1}^{r}(\omega)\right)\right]
\end{aligned}
\]


\section{Numerical test}

Before comparing the results obtained using the generating function
method with those from the scattering method, we first introduce the
scattering method's formula for calculating current
and noise.

\subsection{Scattering method's calcualtion of noise}

Noise is initially expressed in terms of scattering amplitudes. Using
the Fisher-Lee relation, this expression can then be rewritten in
terms of the Green\textquoteright s function and the lead bandwidth.
In this context, we consider a scenario where the central region exhibits
superconductivity(modeled as paring terms at mean-field level). Here,
the indices $i,j$ represent terminals, while $\alpha,\beta$ denote
particle and hole states, respectively.

\begin{equation}
I_{i}=\frac{e}{h}\sum_{\alpha,j\in NS,\beta}\operatorname{sgn}(\alpha)\left[\delta_{ij}\delta_{\alpha\beta}-T_{ij}^{\alpha\beta}(E)\right]f_{j\beta}(E),\label{eq:current sup}
\end{equation}

where $f_{ie}(E)=\left[1+\exp\left(\frac{E-\left(\mu_{i}-\mu_{S}\right)}{kT}\right)\right]^{-1}$,$f_{ih}(E)=\left[1+\exp\left(\frac{E+\left(\mu_{i}-\mu_{S}\right)}{kT}\right)\right]^{-1}$

Noise $\mathcal{S}$

\begin{equation}
\begin{aligned}\mathcal{S}_{ij}=\sum_{\alpha,\beta} & \{\delta_{ij}\delta_{\alpha\beta}f_{i\alpha}(E)\left(1-f_{i\alpha}(E)\right)\\
 & -\text{sgn}(\alpha)\text{sgn}(\beta)\left[T_{ji}^{\beta\alpha}f_{i\alpha}(E)\left(1-f_{i\alpha}(E)\right)+T_{ij}^{\alpha\beta}f_{j\beta}(E)\left(1-f_{j\beta}(E)\right)\right]\\
 & +\delta_{ij}^{\alpha\beta}T_{jk}^{\beta\gamma}f_{k\gamma}(E)-\text{sgn}(\alpha)\text{sgn}(\beta)\left(s_{ik}^{\alpha\gamma*}s_{jk}^{\beta\gamma}f_{k\gamma}(E)\right)\left(s_{il}^{\alpha\delta}s_{jl}^{\beta\delta*}f_{l\delta}(E)\right)
\end{aligned}
\end{equation}

First example a $1\times6$ , hopping $t=1$ simple chain

\begin{figure}[H]
\begin{centering}

\par\end{centering}
\begin{centering}

\par\end{centering}
\caption{Upper Panel: Tight-binding simple hopping chain.Lower Panel: Comparison
of two methods. The current through lead 1 is entirely due to the
transmission from lead 1 to lead 2.}

\end{figure}

\begin{figure}[H]
\begin{centering}

\par\end{centering}
\begin{centering}

\par\end{centering}
\begin{centering}

\par\end{centering}
\caption{Upper Panel: Tight-binding Kitaev chain.Middle Panel: Due to the presence
of the pairing term in the central region, Andreev reflections occur.
As a result, the current is not purely a transmission current and
must be carefully calculated using the scattering method, as described
in Eq.\ref{eq:current sup}.Lower Panel: Comparison of the two methods.}
\end{figure}

\begin{thebibliography}{1}
\bibitem{key-1}A. Kamenev, Field theory of non-equilibrium systems.
(Cambridge University Press, 2023). 2. G.-M. Tang and J. Wang, Physical
Review B 90 (19), 195422 (2014). 

\bibitem{key-2}1. L. S. Levitov and M. Reznikov, Physical Review
B 70 (11) (2004). 

\bibitem{key-3}G.-M. Tang and J. Wang, Physical Review B 90 (19),
195422 (2014). 

\bibitem{key-4}1. J.-S. Wang, B. K. Agarwalla and H. Li, Physical
Review B---Condensed Matter and Materials Physics 84 (15), 153412
(2011). 

\end{thebibliography}

\end{document}
